
\documentclass{article}

% Required NeurIPS package
\usepackage{neurips_2020}

% Other packages the author may need
% \usepackage{amsmath, amssymb, graphicx, etc.}

\title{Your Paper Title Here}

\author{%
  Author Name\\
  Department of Something\\
  University of Somewhere\\
  \texttt{email@domain.com} \\
  % examples of more authors
  \And
  Co-Author Name \\
  Affiliation \\
  Address \\
  \texttt{email@domain.com} \\
}

\begin{document}

\maketitle

\begin{abstract}
This is where the abstract goes. An abstract briefly summarizes the key points of the paper.
\end{abstract}

\section{Background}
What’s the background of the problem? Is it theory-driven, or deeply rooted in some useful application situations? Is it important or necessary? What impact will it bring if you finally solve it?

\section{Definition}
Is there any formal or mathematical definition for your problem? Explain the symbols you may want to use in the proposal writing.

\section{Related Work}
Is your problem a well-established one? If so, review existing approaches and discuss their advantages and disadvantages; if not, survey and describe related problems in the field and list some potentially applicable baseline methods. Remember to cite related work properly.

\section{Proposed Method}
What are the motivations for you to choose it? Which datasets do you propose to experiment on? What baseline approaches do you plan to compare with? How do you implement your proposed method based on the dataset? It is ok to change and improve it later but now try to describe it as detailed as possible.

% References
% \bibliographystyle{plain}
% \bibliography{references}

\end{document}
